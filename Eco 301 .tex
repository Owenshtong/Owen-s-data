\documentclass[10pt]{article}
\usepackage{amsmath}
\usepackage{amssymb}
\usepackage{color}
\usepackage{float}
\usepackage{subfigure}
\usepackage[left=1in,top=1in,right=1in,foot=1in]{geometry}
\usepackage{graphicx}
% \usepackage{tgbonum}
\newenvironment{changemargin}[2]{%
  \begin{list}{}{%
    \setlength{\topsep}{0pt}%
    \setlength{\leftmargin}{#1}%
    \setlength{\rightmargin}{#2}%
    \setlength{\listparindent}{\parindent}%
    \setlength{\itemindent}{\parindent}%
    \setlength{\parsep}{\parskip}%
  }%
  \item[]}{\end{list}}
\renewcommand{\arraystretch}{1.1}
\setlength\parindent{0in}
\pagestyle{empty}

\usepackage{xeCJK} 
\setCJKmainfont{PingFangSC-Light}

\begin{document}

\begin{flushright} 
\end{flushright}


\textbf{Eco 301 Intermediate Microeconomics analysis}

\smallskip

\textbf{中级微观经济学 summer2019}

\normalsize

\medskip

\begin{tabular*}{6.5in}{c}
\hline
\end{tabular*}

\bigskip

\begin{changemargin}{-0.125in}{0in}

\begin{enumerate}

    \item \textbf{Positive analysis: }Develop model that give raise to feasible         statement (with correct data given, it should be testable to be true or false)
        
        \smallskip  
  
         \textbf{Negative analysis: }Produce a description of what should occur (determine what is best)
             
             \begin{itemize}
             	\item Not testable
             	\item Relies on a set of value judgement
             \end{itemize}

            This course focus on positive analysis to model the specific behavior of individuals \& firms.
            
    \medskip
            
    \item \textbf{Consumer Theory}
    
       \smallskip
    
        Modeling the behavior of a individual consumer. Consumer chose a \textit{bundle} (bundle of goods) \& specify a quality of each good in the bundle. Thus a \textit{bundle} is just a point in the commodity space.
    
        \medskip
        
        \begin{enumerate}
    
     \item The symbol $\succeq$ means \textit{at least as good as} and $\sim$ is the negation \textit{not at least as good as}. Then we can include 
         \[
         F \succeq G, \,\, G \nsucceq F \,\, \implies F \succ G
         \]
         \[
         F \succeq G, \,\, G \succeq F \,\, \, \implies F \sim G
         \]
         \textbf{第二行的意思是无法确定哪一个更好 and consequently generate the concept of indifferent set.}
         
         \medskip
         
         \item
         
         \textbf{Axioms and assumption of consumer's choice}
         
         \smallskip
         
          \textit{These axioms can be used to specify whether a set is valid indifference or not}
          
          \begin{enumerate}
          
            \item \textit{Completeness}
               
               \smallskip
               
               Completeness is the short of \textit{preferences are complete}, which means \textbf{people can tell, form any bunch of options, which exactly they prefer}, or \textbf{all bundles are comparable.}
               
            \smallskip
            
            
            \item \textit{Transitivity}
            
            \smallskip
            
            For any bundles F, G and H which are all non-negative, we have
                \[
                F \succeq G, G \succeq H\,\, \implies \,\, F \succeq H
                \] 
                not only $\succeq$, any other operators are possible to be used.
                
                \smallskip
                
            \item \textit{Reflexivity}
                
                \smallskip
                
                For any bundles itself, we have                 
                \[
                 G \succeq G
                \]
                
                \smallskip
                         
                \item \textit{Continuity}
                
                \smallskip
                
                If $G$ is strictly preferred to $H$ (G $\succ$ H), then there must exist another bundle F, s.t F $\succ$ H.
                
                \smallskip
                
                \item \textit{Monotonicity}
                
                \smallskip
                
                More is preferred than less and good provides benefit to consumers at any given quantity. Mathematically can be expressed as: Let G be bundle ($A_G$, $B_G$) and H be bundle ($A_H$, $B_H$). Then $H \succ G$, if 
                
                    \[
                      A_H > A_G \:\&\: B_H > B_G\,;
                    \]
                    \[
                      A_H > A_G \:\&\: B_H = B_G\,;
                    \]
                    \[
                      A_H = A_G \:\&\: B_H > B_G\,;
                    \]
                
                \smallskip
                
                \textbf{Note:} It is possible for $H \succ G$ when one is bigger and another is smaller in the bundle, but that is not because of Monotonicity. 
                   
                    \begin{center}

                    \end{center} 
                
                        
          \end{enumerate}
          
          \smallskip
          
          \item \textbf{Indifference set}
          
          \newtheorem{definition}{Definition}[section]
          
          \smallskip
          
          \begin{definition}
          	Let $S$ =: \{ $G_1$, $G_2$, $G_3$, ....., $G_n$\} be a set of bundles. If 
          	\[
          	\forall \,\, G_i,\, \,G_j \in S,\, G_i \succeq G_j \,\& \, G_j \succeq G_i
          	\]
          	or 
          	\[
          	\forall \,\, G_i,\, \,G_j \in S, \, G_i \succ G_j
          	\]
          	then $S$ is called an indifference set.          	
          \end{definition}
          
          \smallskip
          
            \begin{center}

            Further more, we can extend the indifference set into indifference curve. \textcolor{red}{\textit{All points on the indifference curve provides the same level of utility to consumers.}} What's more, the feature of \textit{completeness} implies that \textcolor{red}{\textit{there is an indifference curve through every possible bundle.}}
            
            \end{center}
          	
          	\smallskip
          	
          	Define a set $\beta_G$ as \textit{the set of bundles that are at least as good as G}
          	
          	%TODO \begin{center}
            
            %TODO \end{center}
          	
          	Then the boundary will be the indifference curve. Then boundary can be any shape. Here are some key features of boundary curve:
          	 
          	   \begin{itemize}
          	   	\item \textbf{Indifference curves are thin} \:By monotonicity, if the curve is not thin, there will be points strictly preferred than another. 
          	   	\item \textbf{Indifference curve never across} \:By transitivity, if two line across, then will concluding a bundle $\succ$ itself.
          	   	\item \textbf{Indifference curve are downward sloping}\: By monotonicity, if not downward sloping, the bundle will be on the curve along with the one that is $\succ$ to it. 
          	   	\item \textbf{}
          	   	
          	   \end{itemize}
          	
         \end{enumerate}
         
         \smallskip
         
         \item \textbf{Convexity}
         
         \smallskip
         
         Preference is said to be \textit{convex} if whenever $x \succeq Y$, then 
         \[
         \lambda x + (1-\lambda)y \: \succeq y, \:\: for\:all\:\:\lambda \in [0,1]
         \]
         \textcolor{red}{some questions remain to be solved}
         
%         \begin{center}
%         	\includegraphics[scale=0.2]{convex}
%         \end{center}
%         
       \begin{figure}[ht]
            \setlength{\abovecaptionskip}{0.cm}
            \setlength{\belowcaptionskip}{-0.cm}
           \centering
         \begin{minipage}[b]{0.4\linewidth}
         \includegraphics[scale=0.25]{convex.png}
         \caption{Convex curve/set}
         \label{fig:minipage1}
         \end{minipage}
         \quad
         \begin{minipage}[b]{0.4\linewidth}
         \includegraphics[scale=0.25]{Indifference.png}
         \caption{Indifference curves}
         \label{fig:minipage2}
         \end{minipage}
    \end{figure}


         \textit{Convex}指两点连线上的点仍在集合内. Points on the curve can be represented as 
         \[
         S =: \{x \in commodity \:space \: \big| x = (\lambda {x_1}_x + (1-\lambda){x_1}_y, \lambda {x_2}_x + (1-\lambda){x_2}_y) \}
         \]
         \textit{Convex} can be \textbf{strictly convex} or \textbf{weakly convex}. All indifference curve must be \textbf{weakly convex.}
         
         \smallskip
         
         We represent consumer's preference by bunch (infinitely) of indifference curves such as figure 2 shown above.   
         
         \smallskip
         
         Notice: 
         \begin{itemize}
         	\item There are infinitely many indifference curves and each bundle have an indifference curve goes through it. 
         	\item Consumers want to chose a bundle from indifference curve as higher as possible $\implies$ maximize personal utility.
         \end{itemize}
         
         \medskip
         
         \item \textbf{Budget constraint}
         
         \smallskip
         
         Given price of $A$ and $B$ and \textit{income level / budget} \textbf{m}, consumer's choice must be feasible/affordable, which is 
         \[
         p_A \cdot A + p_B \cdot B \leq m
         \]
         the budget line becomes
         \[
         B = \frac{m - p_AA}{p_B} = \frac{m}{p_B}  - \frac{p_A}{p_B}A
         \]
         then the \textit{budget set} is the area under the budget line. The slope of budget line is $-\frac{p_A}{p_B}$ (A is the x axis and B is the y axis).
         
         \smallskip
         Notice:
         \begin{itemize}
         	\item  By all stuff discussed above, we can conclude: \textbf{The optimal bundle is at a point where budget line is just touching the highest indifference curve. } 
         	
         	\smallskip
         	
         	\item The slope of the budget line reflect the consumer's willingness to give up one good to get more other good. \textbf{斜率理解为消费者对于放弃购买某种产品来购买另一种产品的意愿。}(\textcolor{red}{Question: Why? Mathmetically the slope is simply the ratio of two goods. If it is true, does it means 消费者更倾向于购买便宜的商品?})
         	     
         	     \smallskip
         	     
         	     \textcolor{red}{Answers: The point that is not the optimal choice does not offer equivalent marginal utility per dolor.我们追求的是单位money带来最大收益,因此在两者带来的money单位收益中最大才停止。不是想要一个good换一个。}  
         \end{itemize}
         
         \medskip
                 
         \item \textbf{Utility}
         
         \smallskip
         
         Measurement of well-beings associated with a bundle. Utility function is 
         \[
         u: \mathcal{R}^2 \longrightarrow \mathcal{R}
         \]
         \[
         (A,B) \longmapsto u(A,B)
         \]
         Consequently we can derive the precise definition of indifference curve as a set of bundles which is 
         \[
         S =: \{x \in commodity\:space \:\big| \: \forall\:i, \:j \in \mathcal{Z}^+, u(x_i) = u(x_j) \}
         \]
         and 
         \[ 
         F \succ G \longleftrightarrow u(A_F, B_F) > u(A_G, B_G)
         \]
         \[
         F \sim G \longleftrightarrow u(A_F, B_F) = u(A_G, B_G)
         \]
         Notice, the number of utility itself make no scene, it works only for comparing two bundle which can bring more utility.
           
           \smallskip
           
           \begin{enumerate}
             \item \textbf{Marginal utility}
             
             \smallskip
             
             Let the utility function be $u(A,B)$. Then the marignal utility of A and B are 
                 \[
                 mu_A = \frac{\partial u}{\partial A}, \:\: mu_B = \frac{\partial u}{\partial B}
                 \]
                 holding another variable fixed. Further more, we should expect 
                 \[
                 \frac{\partial^2u}{\partial A^2} < 0,\:\: \frac{\partial^2u}{\partial B^2} < 0
                 \]
                 because of the diminishing marginal return.
                 
                 \medskip
                 
                 \item \textbf{Marginal rate of substitution (MRS)}
                     
                     \smallskip
                     
                     The slope of indifference curve measure the rate at which the agent is willing to substitute one good for another. The slope is called the \textbf{Marginal rate of substitution (MRS)}, which is 
                     \[
                     MRS = -\frac{dB}{dA} \: \big|_{u(A,B) = constant}
                     \]
                     
                     \smallskip
                     
                     \textbf{The relationship with utility function} 
                     
                     The differential form of the utility function (the total differentiation) can be written as 
                     \[
                     du = \frac{\partial u(A,B)}{\partial A} \cdot dA + \frac{\partial u(A,B)}{\partial B} \cdot dB
                     \]
                     since moving along the indifference curve cause no change in utility, thus
                     \[
                     0 =  \frac{\partial u(A,B)}{\partial A} \cdot dA + \frac{\partial u(A,B)}{\partial B} \cdot dB
                     \]
                     \[
                     \implies -\frac{dB}{dA} = \frac{\frac{\partial u(A,B)}{\partial A} }{\frac{\partial u(A,B)}{\partial B}} = \frac{mu_A}{mu_B}
                     \]
                     which is the ratio of marginal utility of the two goods. Further more, by some algebra, we can get a different form of the equation
                     \[
                     \frac{mu_A}{mu_B} = -\frac{dB}{dA} = \frac{p_A}{p_B} \implies
                     \]
                     \[
                     \frac{muA}{p_A} = \frac{muB}{p_B}
                     \]
                     the above equation suggests that the optimal bundle is where the \textbf{marginal utility per dollar} are equals.
           \end{enumerate}
         
         \medskip
         
         \item \textbf{Demand curve}
          
          \smallskip
          
          By two function 
          \[
          MRS = \frac{\partial u}{\partial A}dA + \frac{\partial u}{\partial B}dB\:\implies A(B,p_A,p_B)\: or \: B(A,p_A,p_B)
          \]
          plug above function into budget constraints 
          \[
          p_AA+p_BB = M
          \]
          we get the demand function
          \[
          A^*(p_A,p_B,M)\: and \: B^*(p_A,p_B,M)
          \]
         
          then we can fix two of the three variables and changes the other one to generate three function --- the demand as a function of $p_A$ or $p_B$ and as a function of income M. Or simply,$A^*(p_A,M)$.
          
          
          
          
          \smallskip
          
           \begin{enumerate}
            \item 
            
            \textbf{Demand as a function of Income}
            Let (A,B) be the bundle and $A(p_A,p_B,M)$. Fix $p_A$ and $p_B$, treat demand of A as a function of demand. We get 
            
            \bigskip
            
            \bigskip
            
            \begin{center}
            	\includegraphics[scale=0.7]{AngleCurve}
            \end{center}
            
            \medskip
            
            With an increasing in income $M$, the optimal choice goes upward. The right side is called the angle curve. So notice, \textit{the point on the engel curve are all optimal choices.}\\
           
            
            Some type of indifference curve can have special shape. Like perfect substitution, whose engel curve is a straight line through the origin. \\
            
            We can also specify $normal$ and $inferior$ good from demand as a function income. Normal good 指\textbf{收入}增加需求增加, inferior good 指收入增加需求减少, which is 
            
             \begin{figure}[ht]
              \setlength{\abovecaptionskip}{0.cm}
              \setlength{\belowcaptionskip}{-0.cm}
                \centering
              \begin{minipage}[b]{0.45\linewidth}
              \includegraphics[scale=0.4]{Nomralgoods}
            
              \label{fig:minipage1}
              \end{minipage}
              \quad
              \begin{minipage}[b]{0.45\linewidth}
              \includegraphics[scale=0.4]{Inferiorfood}
              
              \label{fig:minipage2}
              \end{minipage}
             \end{figure}
            
            \smallskip
            
            $x_1,x_2$ in the first graph are both normal good, however in the second graph $x_1$ is normal good and $x_2$ is inferior good (since with income increasing, the demand decreases). To be more precise, mathematically it becomes
            \[
            if \: \frac{dM}{dx_1} > 0 \:\: \implies \:\: x_1 \:is\: normal \:good
            \]
            \[
            if \: \frac{dM}{dx_1} < 0 \:\: \implies \:\: x_1 \:is\: inferior \:good
            \]
            but notice, the good may change from normal to inferior good as income increases. 
            \medskip
            
            \item \textbf{Demand as a function of Price}
            
            \smallskip
            
            This time fix income M and price of $x_2$, focusing on the change of price of $x_1$. This could generate the real demand curve of $x_1$, cooperating with what we learn from eco101, which is 
            \[
            A\:(p_A,p_B,M)
            \]
            with an increase in price of A, we can derive the demand as a function of price of good itself which is 
            
            \medskip
            
            \medskip
            
            \begin{center}
            	\includegraphics[scale=0.5]{Demands.png}
            \end{center}
            
            \bigskip
            
             Demand curve define quantity as a function of price. To be precise, the quantity consumers are willing to buy at a given price, which is 
             \[
             q(p)
             \]
             and this curve can be represented as 
             \[
             A
             \]
             Inversely, we can define \textbf{inverse demand} as for each amount of demand that consumers are willing to pay, which is 
             \[
             p(q)
             \]
             Further more, we can generate the demand of one good as a function of the other good (4.4 page 54).
           \end{enumerate}
           
           \medskip
           
           \item \textbf{Substitution \& income effect}
              \begin{itemize}
                \item Keeping the real income level fixed at $M$
              	\item Caused by changes in price
              	
              	\smallskip
              	
              	
              	 \textit{Substitution effect} is because of the changes in relative price. Consumers are expected to change to the one which is relatively cheaper
              	 \\
              	 
              	\textit{Income effect} is due to the drop/increase of price, since the real income is fixed, you can by more of the good, then consumer is in effect richer
              \end{itemize}
              
              \smallskip
              Thus, we can \textbf{decompose} the changes caused by changes in price into substitution effect and income effect.
              
                \begin{enumerate}
                  \item 
                  \textbf{Hicksian Method: New price, original utility} 
                  
                  \smallskip
                  
                  New price means new relative price $\implies$ new slope of budget line. 

                  \smallskip
                  
                  Original utility means tangent to the old indifference curve
                  
                  \smallskip
                  
                  \smallskip
                  
                  \begin{center}
                  \includegraphics[scale=0.2]{Hicks}
                  \end{center}
                  The above graph of $x_1$is a case of ordinary good, or further more, a normal good (Ordinary goods consisting of normal and inferior goods. The opposite of ordinary is giffen).
                  \smallskip
                  
                  
                  \item
                  \textbf{Slutsky method: New price, original purchasing power/real income}
                
                \smallskip

                New price means new relative price $\implies$ new slope of budget line. 
                 
                 \smallskip

                Original purchasing power means can exactly afford the original bundle $\implies$ the budget line goes through the original bundle.
                  
                   \bigskip
                   
                   \smallskip

                 \begin{center}
                 \includegraphics[scale=0.3]{Slus}
                 \end{center}
                 
                 \bigskip
                 
                 \smallskip
                 
                 \textit{Substitution effect} is from $x$ to $y$ and \textit{income effect} is from $y$ to z.
                     在实际购买力不变的情况下消费者会去购买Y而不是X了。Here are some notice:
                      \begin{itemize}
                        \item The pivot measures how much demand changes when price changes, keeping purchasing power fixed
                        \item How much will consumer demand (under new price) with just enough budget to buy the original bundle.
                      \end{itemize}
                      
                      \smallskip
                      
                      \textcolor{red}{Why we say hicks is fully compensated? for slustky the consumer can move to a ic with higher utility?}
                      
                      \item
                      \textbf{Summary}
                      
                      \smallskip
                      
                      Either of the two method can be used to determine wither a good is ordinary or giffen good, normal good or inferior good. 
                          \begin{itemize}
                          	\item Compare initial and final budget to determine giffen or ordinary good
                          	\item Compare initial and  intervening bundle to determine normal or inferior. 
                          \end{itemize}
                     Since one is depending on price changes, and one is the determinants of income.
                     
                     \medskip
                     
                     Here are some special cases: 
                     \begin{itemize}
                     	\item 
                     	\textbf{Perfect complement}
                     	  \begin{center}\includegraphics[scale=0.3]{perfect}
                     	  \end{center}
                     	  
                     	  \smallskip
                     	  
                     	  No substitution effect, only income effect (Only possible point to be tangent or going through).
                     	  
                     	  \item 
                     	  \textbf{Perfect substitution}
                     	  \textcolor{red}{Not completed yet}
                     \end{itemize}
                     
                     \medskip
                     
                     What's more, compare the Marshallina and other two demand curve above, Marshalinan's should be more flatter (which a smaller changes) than Hicks's and Slutsky's curve, since the these tow above wipe out the income effects.
                     
                     \begin{center}
                     	\includegraphics[scale=0.8]{SHMvs}
                     \end{center}
                     
                     \smallskip
                     
                      We can consider Hicks and Slutsky substitution effect as an intervening step of going to the Marshallina demand, since Marshallina is the set of bundle that consumers are going to purchase.
                     
                     \smallskip
                     
                     
                  \item 
                  \textbf{Expenditure function}
                  
                  \smallskip

                   We have derived the demand function above, usually use the Marsallian's model by maximizing $u(A,B)$ (of function $v$) function under a given budget constraint, which is what we call \textit{indirect utility function}
                   \[
                   v(p_A,p_B,M) = u(A^*(p_A,p_B,M),B^*(p_A,p_B,M))
                   \]
                   Now on the opposite we consider instead of a certain budget constrain, given a fixed utility, how can we minimize the expenditure to achieve a certain level of utility. Which is 
                   \[
                   minimize\:\:p_AA+p_BB
                   \]
                   since the utility is fixed and price can change, we can think of this indifference curve under given utility as a collection of Hicks's bundles $B^H(p_A,p_B,u)$ and $A^H(p_A,p_B,u)$. Then substitute into the aiming function above we get 
                   \[
                   e(p_A,p_B,u)
                   \]
                   Notice, the function is nondecreasing in price $p$. p increase, e increase in order to achieve the target expenditure minimizing bundle (The increasing e can be observed from the interception with vertical axis). Further more, by the graph above, we can see the intersection of three demand curve is at the beginning optimal bundle. Which mean the solution of \textbf{expenditure minimizing} and \textbf{utility maximizing} are at the same bundle. Then we can derive 
                   \[
                   A^H(p_A,p_B,u) = A^*(p_A,p_B,e(p_A,p_B,u))
                   \]
                   \[
                   A^*(p_A,p_B,m) = A^H(p_A,p_B,v(p_A,p_B,m))
                   \]
                   
                \end{enumerate}
                
                \medskip
                
             \item 
             \textbf{Consumer's welfare \& Consumer's surplus}
                 
                 \smallskip
                 
                 Consumer surplus(CS) is defined as 
                 \[
                 CS = TWP - Amount\:paid
                 \]
              
                 cs equals the total willingness to pay(TWP) - amount paid. This can be expressed as graph (notice this quantity-demand function is derived from \textbf{Marshallian model})
                 
                 \medskip
                  
                 \begin{center}
                 	\includegraphics[scale=0.3]{CS}
                 \end{center}
                 
                 mathematically this can be, let the point to be (q,p)
                 \[
                 CS = \int^0_q\:p(t)dt - p\cdot q
                 \]  
                 then it shows when $p$ increases, $\Delta CS< 0$ and when $p$ decreases $\Delta CS >0$. Then, the \textbf{changes in consumer surplus can be an approximation of changes in consumer's welfare}. 
                 
                 \bigskip
                 
                 \item
                 
                 \textbf{Compensating \& equivalent variation}
                 
                 \smallskip
                 
                 \begin{itemize}
                 	\item \textit{Compensation variation} \:\:is the amount of income must be given to or taken away from the consumer to achieve the original utility. 价格改变 但通过补偿使保持原来utility水平.
                 	     \begin{center}
                  	     \includegraphics[scale=0.4]{CV.png}
                         \end{center}
                 	\item \textit{Equivalent variation} \:\: is the amount of income that must be given or taken away from the consumer to achieve new utility level that can be brought up by a change of price. 价格不改变 但通过增加或者拿走income来达到价格改变能带来的相同level of utility.
                  
                  \begin{center}
                  	\includegraphics[scale=0.4]{EV.png}
                  \end{center}
                  
                 \end{itemize}
                 
                 The changes of intersection with vertical axis can only be effected by income changes (the parallel shift), since the price of B ($x_2$) remains constant. Thus, if we want to assign a dollar value of $EV$ and $CV$ we just need to multiply the change by $p_B$, since the difference of the EV and CV is 
                 \[
                  \frac{M_1}{p_B} - \frac{M_2}{p_B}
                 \]
                 the subscript indicates the original and new budget line.\\
                 
                 Further more, we can build the relationship with 
                 demand function. Since no matter $EV$ or $CV$, the bundles are moving on the same indifference curve. Thus we can use the Hicksian demand function to obtain graphically 
                 
                 
                 \begin{figure}[H]
                   \centering
                   \subfigure[Compensating variation]{
                   \begin{minipage}[t]{0.25\linewidth}
                   \centering
                   \includegraphics[scale=0.45]{CVS}
                   %\caption{fig1}
                   \end{minipage}%
                   }%
                   \subfigure[Equivalent variation]{
                   \begin{minipage}[t]{0.3\linewidth}
                   \centering
                   \includegraphics[scale=0.45]{EVS}
                   %\caption{fig2}
                   \end{minipage}%
                   }%
                   \subfigure[Consumer surplus]{
                   \begin{minipage}[t]{0.25\linewidth}
                   \centering
                   \includegraphics[scale=0.4]{CS2}
                   %\caption{fig2}
                   \end{minipage}
                   }%
                  \end{figure}
                  
                  \textcolor{red}{Stop discussing here: the reason why this is true is explained by Shephard’s Lemma which is not covered.}
                  
                  
                  \medskip
                  
                  Mathematically, the above result can be written as 
                  \[
                  CV = \int^{p_A'}_{p_A} A^H(t,p_B,u_0)dt\:\: and \:\:EV = \int^{p_A'}_{p_A} A^H(t,p_B,u_F)dt
                  \]
                  The area is to the left is because the dependent variable is on horizontal axis. \textbf{Notice: the computation is true only when the price of vertical good is 1, since only under this case the difference in intersection is exactly the change of income.}


                  \bigskip
                  
                  \item 
                  \textbf{Price index}
                  
                  \smallskip
                  
                  The price index usually fix the quantity of goods, or just fix a bundle of good and comparing the cost of purchasing the two bundles in two times period. Quantity index is just the opposite, which is fix the price and compare the quantity consumed in two period.
                  
                  \begin{enumerate}
                  
                  \item \textbf{Laspeyres price index}
                  
                  \smallskip
                  
                  Let the superscript indicate the time period and the subscript be the type of good. Then, for the case with only two type of good $A$ and $B$, the Laspeyres price index is 
                  
                  \[
                  L^p=\frac{p_A^1A_0+p_B^1B_0}{p_A^0 A_0+p_B^0B_0}
                  \] 
                  and for a more general case which is the bundle is consisting of $n$ types of good at a certain time $t$
                  \[
                  L^p = \frac{\sum^n_{i=1}p^t_iq_i^0}{\sum^n_{i=1}p^0_iq^0_i}
                  \]
                  \textcolor{blue}{there is a more technical explanation of this index in eco301 document file on page 6.}
                  
                  \medskip
                  
                  \item 
                  \textbf{Paasche price index}
                  
                  Still using the notation in (a). The difference is instead of using old bundle, we use the current or new bundle and changing prices, which is 
                  \[
                  P^p = \frac{p_A^1A_1+p_B^1B_1}{p_A^0A_1+p_B^0B_1}
                  \]
                  usually new period situation over old situation. 
                  
                  \medskip
                  
                  \item
                  \textbf{Ideal price index}
                  
                  The ideal price index is trying to compare expenditure required to reach given utility in two different time period. Consider the expenditure function
                  \[
                  e_A(p_A,p_B,u) = p_A\cdot A^H(p_A,p_B,u) + p_B\cdot B^H(p_A,p_B,u)
                  \]
                  Let's say consumer achieve $u(A_0,B_0) = u_0$ at time $t_0$ (maximizing utility) and also at $t_1$ the utility is still $u_0$. Then the ideal price index is defined as 
                  \[
                  I_p=\frac{e(p_A^1,p_B^1,u_0)}{e(p_A^0,p_B^0,u_0)}
                  \]
                  
                  \textcolor{red}{Things needed to be made up.}
                  
                  \end{enumerate}
                  
                  \bigskip
                  
                  \item 
                  
                  \textbf{Endogenous Income}
                  
                  \smallskip
                  
                  Instead of having a "constant" income level $m$, the income of a consumer is measured by the initial value of their endowment, and there is no exogenous income. So if the initial bundle is $(\overline{x}, \overline{y})$, then the consumer's budget constrain is 
                  \[
                  p_x\cdot x + p_y\cdot y \leq p_x\cdot \overline{x} + p_y\cdot\overline{y}
                  \]
                  Cases are very possible that the endowment bundle is not the optimal choice s.t consumers can sell one good and exchange for another good to achieve the optimal. Mathemetically, let $E(A^E,B^E )$ be the endowment and $O^*(A^*,B^*)$ be the optimal choice, the optimal bundle is subject to the endowment, which is 
                  \[
                  p_A\cdot A* + p_B\cdot B^* = p_A\cdot A^E + p_B \cdot B^E
                  \]
                  then rearranging  we get 
                  \[
                  p_A(A^*-A^E) = p_B(B^E-B^*)
                  \]
                  LHS is value of B sold and RHS is the value of A purchased (or the other way around). Consider the Marshalian demand function, we use the endowment to replace the fixed income level $M$, which is 
                  \[
                  A^*(p_A,p_B,A^E,B^E) \: \& \: B^*(p_A,p_B,A^E,B^E) 
                  \]
                  we can call this the \textbf{Marshalian demand with endowment}. So if $A^*-A^E < 0$, it means we are selling $A$ for $B$ or using the difference of B. To analyze this case, we need to introduce a new concept, \textbf{Endowment effect}.
                  
                  
                  
                  
                   
%                   \textcolor{blue}{Some more discussion about three type of demand curve: }
%                  
%                  \begin{enumerate}
%                    \item \textcolor{blue}{When there's no income effect, the Hicksain and Slusky demand curve become the same. }
%                  	\item \textcolor{blue}{Quasi-linear utility has no substitution effect. Thus its Hicksin and Slustky demand curve are the same.}
%
%                  \end{enumerate}

                  \medskip
                  
                  \begin{enumerate}
                    \item \textbf{Model: Consumption \& saving model}
                    
                    \smallskip
                    
                    Let $W_1 and \:W_0$ denote the wealth a consumer held at time $t=1 and \:t=0$. We can think of $W_0$ as the current consumption and $W_1$ as the future consumption, or the saving of a consumer comparing to the current point of time. The indifference curve shows the trade-off between consumption this year and next year. \\
                       
                        Let $E=(W_0^E,W_1^E)$ be the endowment and the interest rate is $r$. If we set aside 1 dollar at $t=0$, we will get $(1+r)$ at $t=1$. Thus the slope of the 'budget line' should be $-(1+r)$. Then the endowment budget line become 
                        \[
                        W_0(1+r) + W_1 = W_0^E(1+r) + W_1^E
                        \]
                       both LHS and RHS are the future value of the consumer's choice and the equation suggests that the all future value of the choices are subject to the endowment. Specifically for the optimal choice we have 
                        \[
                        W_0^*(1+r) + W_1^* = W_0^E(1+r) + W_1^E
                        \]
                        Not only for future value, the present value are also equivalent which is 
                        \[
                        W_0+W_1\cdot v = W_0^E + W_1^E \cdot v
                        \]
                        where $v$ is the discount factor. Graphically, it is 
                        
                        \begin{center}\includegraphics[scale=0.4]{Endowment}
                        \end{center}
                        
                        Staring from the endowment, moving along the arrow upward means the consumer is \textbf{lending out money or saving} to put off consumption and moving downward along the arrow means \textbf{borrowing money}.\\
                        \\
                        Now if the interest rate changes, just like the price changes in previous case, since now we have endowment instead of fixed income, the budget line with new interest rate will still across the endowment bundle and turns around. \textcolor{red}{things needed make up, question remaining. Model: Firm decision making}
                        
                        \bigskip
                        
                        \item \textbf{Model: Labor or Leisure -- allocation of time}
                        
                        \smallskip
                        
                        This model is trying to describe the relationship between leisure and working with scarcity of time. First introducing some notations: 
                        
                        \smallskip
                        
                        \begin{itemize}
                        	\item $w$: \textit{Wage rate} $    $Dollar paid to individual per unit of time
                        	\item $p$: \textit{Price of Consumption Good} $    $ Here we treat all goods consumed as an integrated unit by sth like price index etc. 
                        	\item $L$: \textit{Leisure}$    $ Total time spent on leisure in a certain time interval
                        	\item $c$: \textit{Consumption} $    $ Quantity of the 'integrated' good consumed.  
                        	\item $T$: \textit{Total hours per time interval} If we set the time interval to be a day, then T should be 24h
                        \end{itemize}
                        
                        \smallskip
                        
                        Comparing the case with fixed dollar valued income, now our constrain become time and its corresponding wage rate which is $wT$. Thus we can generate out budget line as 
                        \[
                        wL + pc \leq wT
                        \]
                        
                        and therefore the budget line becomes 
                        \[
                        wL + pc = wT
                        \]
                        
                        it means, \textbf{we divide our total time into leisure and working, and spent all the wages generated form working to consume goods.} Also, we assign a dollar value on \textbf{time} by giving a wage rate. Or, we can consider leisure as a special commodity with price $w$ which need to purchase.\\
                        
                        Further more, the $wage \: rate$ can be considered as the opportunity cost of leisure. Individual lose $w$ if he takes on a unit time of leisure. In this model we assume the price is just 1 dollar. \textcolor{red}{Needed to be made up. Note is May21.}
                        
                        
                          
                          
                        
%                        \textcolor{blue}{Question: Do we assume there's not money remaining after t=0, or we assume consumer uses up their wealth?}
                        
                  
                  \end{enumerate}
                  
                  \bigskip
                  
                  \item 
                   
                  \smallskip
                  
                  \textbf{Choice under uncertainty}
                 
                  \medskip
                  
                  The whole discussion is based on \textbf{von-Neumann Morgenstern} principles, consisting of a set of axioms (\textcolor{red}{maybe made up here}). The utility function is no longer ordinal but also not fully cardinal. So the transformation used in previous section are no longer applicable. 
                  
                  \medskip
                  
                  Let $W$ indicates the wealth and $b$ and $g$ represent bad and good states respectively. Let $\pi$ be the probability. Von-Neumann Morgenstern theorem define the utility function as 
                  \[
                  u(W_b,W_g) = \pi_b \cdot u(W_b) + \pi_g \cdot u(W_g) = \pi_b \cdot u(W_b) +  (1-\pi_b) \cdot u(W_g)
                  \]
                 
                  Now consider the indifference curve. \textcolor{red}{need more made up here. Review refer to notes}
                 
                 
                 \bigskip
                 
                 \item \textbf{Producer Theorem}
    
    \medskip
    
    Firms and agent makes producing decision. Given tech., decide quantity to produce.
    \\
    
    \begin{enumerate}
        \item \textbf{Production function}
        \\
        \\
        Maximum amount of goods can be produced by given quantity of set of input. Usually we use $L$ as \textit{labor}, $K$ as 
        \textit{capital} and $T$ as \textit{land} and $E$ as \textit{energy}. Then the production function becomes
        \[
        y= f(x_1,x_2,x_3,..,x_n) 
        \]
        It is determined purely by technology,nothing about values. For the case of tow type of input $L \& K$, we have 
        \[
        y = f(L,K)
        \]
        and the partial derivative $\partial L$ and $\partial K$ are marginal production of labor (MRL) and marginal production of capital (MRC) respectively. Generally assume MRi are greater than zero which means free-disposed of unnecessary input. 
        \\
        \\
        Similar to indifference curve in consumer's theorem,instead of making utility as individual, the counterpart in producer's theorem is the level of output. The "indifference curve" is called \textbf{\textit{isoquant curve}} here, which means all those kind of combination of input on the curve will bring out the same amount of output. 
        \\
        \\
        The isoquant curve can be types of indifference curve (e.g perfect substitution, complement, Cobb-dogolus ect.). \textcolor{red}{Labor intensive and capital intensive made up here}. The isoquant is \textbf{\textit{cardinal}} but not ordinal, it represents the real amount of production of given input. So the order-keeping transformation do not make sense here. The Cobb-dogolus isoquant is like
        \[
        y=f(L,K) = A\cdot L^\alpha K^\beta
        \]
        where the parameter $A$ is called the \textit{technology parameter} or \textit{total factor productivity}.
        \\
        \\
        \textit{Slope of isoquant curve} is called the\textit{marginal rate of technical substitution.} It is derived by the same mean in consumer theorem. Let $y=f(l,K)$, then moving along the isoquant curve we have the total derivative 
        \[
        0 = \frac{\partial f}{\partial L} \cdot dL + \frac{\partial f}{\partial K} \cdot dK \implies \frac{dK}{dL}= - \frac{MRL}{MRK}
        \]
        notice here we put L on horizontal and K on vertical axes. 
        
        \medskip
        
        \item \textbf{Return to (of)  scale}
        \\
        \\
        \textit{Definition}\,\,\,How production varies as all input scaled proportionately. Mathematically, let $\lambda > 1$, it can be 
        \begin{itemize}
            \item if $f(\lambda L, \lambda K) > \lambda f(L,K)$, then we have increasing R.T.S
            \item if $f(\lambda L, \lambda K) < \lambda f(L,K)$, then we have decreasing R.T.S
        \end{itemize}
        
        For the Cobb-Dogolus isoquant curve, the return to scale is lik this. Since we are comparing the situation above, we can get 
        \[
        f(\lambda L,\lambda K)= A\cdot \lambda^{\alpha+\beta} \cdot L ^\alpha K^\beta 
        \]
        comparing with the $\lambda f(K,L)$ we can get if $\alpha+\beta = 1$, the return to scale is constant, if $<1$ decreasing and if $>1$ is increasing. 
        \textcolor{red}{A graph should be included here to show the return to scale.}
        \\
        
        \item \textbf{Short run \& Long run}
        
        \medskip
        
        In short run, at least one type of input is fixed, while in long run any input is varied. The case for shor run can be represented as 
        \[
        y=f(L,\Bar{K}) = f(L)
        \]
        or the other way around.The curve is also convex (concave down) because of the diminishing marginal return. This is also indicated as \textcolor{red}{Graph needed and check with prof the isoquant curve should be further from each other as L increases}. 
        \\
        
        \item \textbf{Decision making}
        \\
        \\
        Firms purpose is to maximize profit. It has two approach. When a certain level output/purpose is given or establish, the firm need to minimize the cost of input, or when a certain quantity of input is given, it needs to maximize output.The following discussion is under perfect competitive market in which each firm is only a very small part of the market.
        \\
        \\
        Let $p$ be price of output determined by the market, $w$ be the price of labor and $r$ be the price of capital input. If the firm use only labor and capital to produce, then with a given production function we have 
        \[
        \pi (L,K) = p\cdot f(L,K) - (w\cdot L + r\cdot K)
        \]
        to maximize $\pi(L,K)$ just let 
        \[
        \frac{\partial \pi}{ \partial L} = \frac{\partial\pi}{\partial K} = 0
        \]
        the solve for K and L. This is called a \textbf{\textit{unconstrained maximum}}.By this we can get the \textbf{\textit{factor demand}} function of L and K
        \[
        L^*(p,w,r)\,\,\,and\,\,\,K^*(p,w,r)
        \]
        
        \textit{Short run case}\,\,\, In short run, let us fix one input $K$. Then the profit function becomes
        \[
        \pi^{SR}(L,\Bar{K}) = \pi(L) = p\cdot f^{SR}(L,\Bar{K}) - (w\cdot L + r\cdot \Bar{K})
        \]
        this can be observed from the intersection of $\pi^{SR}$ curve with the vertical axes and intersection of $f^{SR}$ with vertical axes. The intersection of $\pi^{SR}$ indicates the initial fixed cost. To maximize the SR profit, just need
        \[
        \frac{d\pi}{dL} = P \cdot MPL -W = 0 \implies \frac{W}{P} = MPL
        \]
        Then rephrasing the above equation, we get 
        \[
        f^{SR}(L,\Bar{K})= y(L,\Bar{K}) =\frac{\pi^{SR}+r\cdot\Bar{K}}{p}+ \frac{w}{p}\cdot L
        \]
        The slope is exactly satisfy the case when $\pi^{SR}$ is maximized. Thus the above $y$ curve is the optimal. With the given level of profit, the line 
        \[
        y^*(L)= \frac{\pi^{SR}+r\cdot\Bar{K}}{p}+ \frac{w}{p}\cdot L
        \]
        is called the \textbf{\textit{isoprofit line}}. \textcolor{red}{A graph needed here to show the isoprofit curve.} The logic here is, we know at the optimal point the slope of $y^{SR}$ curve equals $\frac{W}{P}$ and there must be a point satisfy this. The tangency is given by $y=f(L)$ which is the technology determinants and the isoprofit curve.
        \\
        \\
        Now with the profit function the firm cam make production decision.
        \begin{enumerate}
            \item \textit{Case one}\;\;\; At optimal choice $L^*$, if  $\pi^{SR} > 0$, the firm should produce positive output at corresponding $y^*$.
            \\
            \item \textit{Case two}\;\;\; At optimal choice $L^*$, if  $\pi^{SR} < 0$, the decision depends one the nature of fixed cost $\Bar{K}$: 
         
         \smallskip
         
            \begin{itemize}
                \item If $\Bar{K}$ is sunk cost which is not recoverable, the firm should keep producing as long as the revenue can cover the variable input $L$. That is $py^* > wL$ 
                
                \smallskip
                
                \item  If $\Bar{K}$ is not a sunk cost which means it can be recovered, then it is better to sell all equipment and produce noting.That is $L = 0$, $K = 0$ and consequently $\pi = 0$ with $y= 0$.
            \end{itemize}
            \textcolor{red}{Need make up here.}
        \end{enumerate}
        
        Another approach. Instead of giving input, now we are given a fixed level of output we need to achieve. So in order to maximize profit, we need to minimize our cost. Let $y=f(L,K) = \Bar{y}$, we want to
        \[
        Min \;\; wL + rK = \mathcal{C}
        \]
        where C indicates the total cost. Rearrange the above equation we get 
        \[
        K = \frac{C}{r}- \frac{w}{r}L
        \]
        the curve is called \textbf{\textit{isocost line}}. The given level of output offers a particular isoquant curve, we just need to find the lowest isocost line which is tangent to the isoquant curve \textcolor{red}{Graph needed.}At the optimal choice point we have 
        \[
        -\frac{w}{r} = \frac{MPL}{MPK}
        \]
        Then combine the above equation with $y=f(L,K) = \Bar{y}$, we can derive the \textbf{\textit{conditional factor demand}} function which are 
        \[
        L^c(w,r,\Bar{y})\,\,\&\,\,K^c(w,r,\Bar{y})
        \]
        substitute above equation into purpose function we get the cost function
        \[
        \mathcal{C}(w,r,y) = wL^c + rK^c 
        \]
        usually use $q$ instead of $y$ and leave out $w$ and $r$. The formal cost function becomes just $\mathcal{C}(q)$.
        It means the minimal cost needed to produce $q$ quantity of output. This cost function hold in any market structure. 
        \\
        \\
        The total cost of a firm can also be separated into fixed and variable cost. Fixed cost does not vary as $q$ increases while variable cost does. So the total cost function can be deconstructed as
        \[
        \mathcal{C}(q) = F + VC(q)
        \]
        and divided by quantity $q$on both side we get the average cost fucntion 
        \[
        AC(q)=AFC(q) + AVC(q) =\frac{F}{q} + AVC(q)
        \]
        and marginal cost is 
        \[
        MC(q) = \mathcal{C}'(q) =VC'(q)
        \]
        and the marginal cost curve cut AC and AVC curve at their minimum points.\textcolor{red}{Graph needed.}Economic scale applied here means lower average cost as output increases. This is also the reason why the AC function is of U shape. Before the minimum point. \textbf{\textit{local eco of scale exists}} and after the minimum point the eco of scale is exhausted.
        \\
        \\
        \item \textbf{Neutral monopoly}
        \\
        Industry where the cost of production are minimized by having only one firm produces, two or more firm will increase costs.\textcolor{red}{need to be made up}
        \\
        \item \textbf{More general approach to profit maximization}
        \\
        Still after the first step we get the cost function $\mathcal{C}(w,r,\Bar{y})$, but now we do not assume competitive market and consider a case for any market structure. Let $\mathcal{R(q)}$ be the revenue function. The profit function becomes 
        \[
        \pi(q) = R(q) - \mathcal{C}(q)
        \]
        The revenue function can be in any form (e.g. $\mathcal{R}(q) = p-q, p(q) - q$ etc.). Now we want to maximize the above equation
        \[
        MAX \;\;\; \pi(q) \;\;\;\implies\;\;\;\; \frac{\partial \pi}{\partial q} = 0 \;\;\;\implies\;\;\;\mathcal{R}'(q) - \mathcal{C}'(q) = 0
        \]
        \\
        \textit{Firm supply in competitive market}
        \\
        \\
        Firms regard themselves as very small relative to the market and thy can produce any amount they want with no influence on market price. So some conclusions are 
        \begin{itemize}
            \item \textit{Market price $p^*$ equals the marginal revenue}
            \item \textit{Market price $p^*$ equals the average revenue}
        \end{itemize}
        and firms will always produces until marginal revenue (or here the market price $p^*$) equals marginal cost which is the profit maximizing choice. \textcolor{red}{Graph needed refer to the graph on x+2}. Thus 
        \begin{itemize}
            \item \textit{The supply curve is exactly the same as marginal cost curve}.
        \end{itemize}
        \textcolor{red}{Graph insert here}
        here are three cases: 
        \begin{itemize}
            \item  When price $\Tilde{p} > \Bar{p}$ then produce at corresponding $\Tilde{q}$ $\implies$ $q^s(p) >0$ $\pi >0$ for sure
            \item  When $\Tilde{p} < \underline{p}$ produce nothing since the marginal revenue can not even cover the variable input $\implies$ $q^s(p) >0$ and $\pi <0$ for sure. 
            \item  When price $\underline{p}<\Tilde{p}<\Bar{p}$ depends on nature of fixed cost (discussed before).
        \end{itemize}
        Notice the $MC(q)$ is a function of quantity. 
        \bigskip
        
        \textit{Market(industry)/total supply in competitive market}
        \\
        \\
        The total supply of a certain product is just the integrated output of all firms that are producing it. Here we use the inverse of $MC(q)$ written as $Q(p)$ which means the total quantity produced under a given market price. \textcolor{red}{Graph insert here}.
        \\
        (\textit{The supply curve can be expressed as a function either $q(p)$ of $p(q)$ and in the competitive market the price is exactly the marginal revenue MC and the supply curve are the same.})
        \\
        \\
        \textit{Market demand}
        
        \bigskip
        
        Similar to market supply, market demand is the aggregation of all individual demand to a certain product. We know from consumer's theorem the Marshalian demand of a person is a function $A_i(p_A,p_B,M_i)$. So typically we need to add the up and the total demand becomes
        \[
        A_t(p_A,p_M, M_1,M_2,...,M_n) = \sum^n_{i=1}A_i(p_A,p_B,M_i)
        \]
        In order to simplify the aggregation function, we use the function 
        \[
        A_t(p_A,p_B,\sum^n_{i=1} M_i)
        \]
        to approximate. But this depends on some very strong assumptions: We assume the existence of \textit{many consumers and sellers(supplier)}, \textit{homogeneous goods (perfect substitution)}, \textit{free entrance and exists} and \textit{perfect information}. \textcolor{red}{Graph insert here with a regular equilibrium graph showing excess demand and supply}.
        \\
        \\
        \item \textbf{Elasticity}
             Elasticity is used to measure sensitivity of one variation to another. The main formula for all kinds of elasticity is 
        \[
        E = \frac{\%\Delta Var1}{\%\Delta Var2} = \frac{\frac{dV_1}{V_1}}{\frac{dV_2}{V_2}} = \frac{dV_1}{dV_2}\cdot\frac{V_2}{V_1}
        \]
        which is the ratio of the change in percentage. Here are some common elasticity used: 
        \begin{enumerate}
            \item \textit{Own-Price elasticity of demand:}
            Let the demand function of A be $A(p_A,p_B,M)$. Then the own price elasticity is 
            \[
            \mathrm{E}^D_{A,p_A} = \frac{dA}{dp_A}\cdot \frac{p_A}{A}
            \]
            Notice the dependent variable (Quantity demand) is on the horizontal axes and the independent one is one the vertical (Price). For own price elasticity, if the $|E^D_{A,p_A}| <1$, it is price inelastic demand and elastic price demand if $1 < |E^D_{A,p_A}|$.
            \\
            \item \textit{Cross-price elasticity of demand:}
            Let the demand function of A be $A(p_A,p_B,M)$. Then the cross price elasticity of demand is 
            \[
            E^D_{A,p_B} = \frac{dA}{dp_B} \cdot \frac{p_B}{A}
            \]
            Notice cross means that it measures the variation of quantity demand of good A to price of another good B. So it is A over $p_B$.
        \end{enumerate}
        If a demand is elastic it indicates that a smaller decrease in price can generate a bigger increase in quantity demand which can consequently causes a bigger increases in total revenue. \textcolor{red}{A math proof needed. } Also from the part of derivative of elasticity we know, the flatter the more elastic and the more steeper the more inelastic (but not always true).
        \\
        \\
        In the similar way we can define the elasticity of supply. \textcolor{blue}{We haven't formulate a supply curve in general case} It is 
        \[
        \eta^S_{A,p_A} = \frac{dq^S}{dp_A}\cdot \frac{p_A}{q^S}
        \]
        
        \medskip
        
        \textit{Profit Maximization in monopoly}\;\;\;\; No matter in what kind of market, MR = MC will bring the maximal profit. For monopoly, the MR is not constant (which is price determined by market in perfect competitive market). So we generate 
        \[
        TR(q) = p(q)\cdot q  
        \]
        where $p(q)$ is the inverse demand. Then the marginal revenue is 
        \[
        MR(q) = TR'(q) = p'(q)q+p(q)
        \]
        For maximization we want
        \[
        MR(q) = MC(q)  \implies p'(q)q+p(q) = MC
        \]
        what's more, we know the price index for demand here is 
        \[
        \mathrm{E}^D = \frac{dq}{dp}\cdot\frac{p}{q}
        \]
        combine the two equation together 
        \[
        MC = p(1 - \frac{1}{|\mathrm{E}^D|})
        \]
        Consequently, we derive the \textbf{\textit{Lerner index}} which is a measurement of market power of monopoly. \textcolor{blue}{More elastic, less market power.}
                                
                                   	
                  \end{enumerate}
                  
                  \bigskip
                  
                  \item 
    \textbf{Monopoly and price discrimination}
    
    \medskip
    
    \textit{Uniform pricing}\,\,\,Exactly same price is charged on every consumer
    \\
    \\
    \textit{No-linear pricing}\,\,\,Price arrangement where average price paid per unit is not constant w.r.t number of unit.
     \\
     \\
     \textit{Price discrimination}
     \begin{itemize}
         \item Price arrangement where different consumer pay different price
         \item monopolist (or other firms with market power) may increase profit by using sth other than uniform pricing.
     \end{itemize}
     
     \begin{enumerate}
         \item \textbf{Third degree price discrimination}
         
         \medskip
         
         In monopoly, monopolist knows exactly the demand curve of the market and they divided consumers into different groups (e.g high demand \& low demand group, business \& private users, working adult \& students or services). It is known that \textit{different group are willing to pay different price for products.} 
         \\
         \\
         We have two assumption for third degree discri.: 
         \begin{itemize}
             \item Monopolist can tell which group consumers belong to;
             \item Monopolist can prevent resale of items between different groups.
         \end{itemize}
         Under third discrimination, monopolist set two different prices $p^H$ and $p^L$ for high and low demand group respectively. This is called a \textit{uniform monopoly price (linear price)} which means people in the same group facing exactly the same price of the good.
         \\
         \\
         Price rule is: Group with \textit{lower price elasticity} should be charged with \textit{higher price} and group with higher demand should be charged with lower price. \textcolor{red}{need make up. Graph neeeded to show on Jun. 11th page 2 note}
         \\
         \\
         From the $3^{rd}$ degree discrim. we have 
         \[
         \pi(q^H,q^L) = p^H(q^H)\cdot q^H + p^L(q^L)\cdot q^L - \mathcal{C}(q^H,q^L)
         \]
         \[
         = p^H(q^H)\cdot q^H + p^L(q^L)\cdot q^L - \mathcal{C}(q^H+q^L)
         \]
         then for profit maximization we have 
         \[
         MR^H(q^H) = mc(q^H+q^L)\,\,\, \& \,\,\, MR^L(q^L) = mc(q^H+q^L)
         \]
         which is shown on the graph above \textcolor{red}{specify the point} \textcolor{blue}{Why?} \textcolor{red}{graph on page4 needed} 
         \\
         \\
         With $3^{rd}$ price discrim., profit to monopolist will be larger than uniform price of monopoly and thus $\Delta PS > 0$. For consumers, $CS$  decreases since high demand group is charged higher price ($\Delta CS^H =0$ if $p^m=p^H$), $CS^L increases$, thus in general $\Delta TS \geq 0$.\textcolor{blue}{Do not understand at all}
         \\
         \\
         \item \textbf{Non-linear pricing}
         
         \bigskip
         
         Model here is the \textit{two-tariff} model. As the name suggests, two parts includes 
         \begin{itemize} 
             \item  \textit{Fixed/access charges} A\;\;\; Like a entrance or purchasing permission, consumers need to pay to gain an access.
             \item \textit{per-unit charges} p\,\,\, Regular price fee paid on each item purchased.
         \end{itemize} 
         the total cost is $= A + p\cdot q$ and average total cost $ = \frac{A}{q} + p$ (decreasing as p increases). \textcolor{blue}{not important for exam. A detailed ppt is saved to pc for reference}
     \end{enumerate}
    
    \bigskip
    
    \item \textbf{Government intervention in market (Taxes \& Subsidies)}
     
     \medskip
     
     \begin{itemize}
         \item \textbf{Tax}\;\;\;Amount paid to government for each unit produced and purchased
         \item \textbf{Subsidy}\;\;\;Amount paid by the government for each unit provided and purchased
     \end{itemize}
     With tax and subsidy, there is money paid by consumer $p^d$ from demand side and money received by producer $p^s$ from supply side.
     \\
     \\
     Let tax be fixed for each unit, then 
     \[
     p^d_t = p^s_t + t\,\,\,\&\,\,\,p^s_s = p^d_s + s
     \]
     notice the demand and supply price on both side. We can also have tax and subsidy defined by percentages which is called the \textit{volume tax} 
     \[
     p_\tau^d = p_\tau^s(1 + \tau) \,\,\,\&\,\,\, p^s_s=p_\tau^d(1+s)
     \]
     \textcolor{red}{two graph needed here to show the tax, subsidy and dead-weigh lost}
     \\
     A feature of the tax is, no matter who buy this product, he need to pay the same amount of taxes on it, so it is not like the consumer's surplus, different people have different evaluation on it and then surplus exists. Similar to subsidy.
     
     \bigskip
     
     \item \textbf{incidences of tax \& subsidy}
     
     \medskip
     
     How tax and subsidies changes price to each party. Incidences of consumers and producers are 
     \[
     \frac{p_t^d - p^*}{t} \,\,\,\&\,\,\, \frac{p_*-p_t^s}{t}
     \]
     they are related to relative elasticity of demand. \textcolor{red}{graph needed on page7} by this we can see tax has a greater impact on consumer or producers. DWL is smaller when demand and supply are very inelastic.
     \\
     \\
     Still assume there are two goods A and B. Now the problem of taxation becomes how to reach a certain level of target tax amount and maximize the the TS or minimize the DWL. This is 
     \[
     \max \,\,\,TS_A + TS_B 
     \]
     which is the aggregate total surplus, or minimize
     \[
     \min\,\,\, DWL_A + DWL_B
     \]
     both equation should be subject to the target level of government revenue $G_A + G_B = \Bar{G}$. This is called the 
     \textit{Ramsey taxation solution} or the Ramsey taxes. The main ideas is to \textit{Set higher taxs on products with less elastic demand and then generate less dead-weigh loss.}
                  
                  \end{enumerate}
                  
                  
                  
                 
\end{changemargin}

\end{document}